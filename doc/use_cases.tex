
The utimate goal of this standard is to allow clients to get a better understanding of the data complexity whether it is legacy or set for a particular model. It is however important not to force clients to choose between a regular data processing and a full object approach.
The design of the mapping syntax addresses the necessity of supporting intermediate levels of use that correspond to concrete use cases.

\begin{itemize}
  \item The recent modelling efforts (Meas, Coords and PhotDM) provide very accurate descriptions of coordinate frames or photometric 
        systems that need to be serialised in VOTables:
  \begin{itemize}
    \item Clients often need to get such coordinate systems descriptions upon different axes to properly display datasets. 
    \item Data sets can contain different quantities expressed in the same coordinate system (corrected position vs raw position) or 
          quantities of the same nature but expressed in different systems (sky coordinates in ICRS vs Gal). 
          This is partially supported by the actual VOTable schema (TIMESYS, COOSYS) but any change in these ad-hoc mapping 
          elements or the addition of any other elements (e.g. X-ray energy band) would require VOTable version upgrades. 
          Using the mapping syntax diconnects the level of description of the different axes from the VOTable schema.           
  \end{itemize} 
  
  \item Quantities made with multiple components (spread over multiple table columns) may need to be aggregated by the client to be used correctly:
  \begin{itemize}
    \item value-error associations
    \item error quantities split in several columns (covariance matrices). 
    \item quantities with quality flags
    \item positions with proper motions and/or parallax to compute error ellipses or positions at a given epoch
  \end{itemize} 

  \item The above cases relate to the description of instances in a
  single VOTable. Model annotations become even more important in cases
  where interoperability is required, for example to identify and
  reconcile quantities from multiple VOTables.

  This can be achieved by giving  common data structures for all 
  quantities of interest. This is the purpose of e.g. Measure model 
  which proposes classes for most of the physical quantities that can 
  be rendered by the mapping syntax. Measure classes are not meant to 
  be used as standalone elements but as parts of host models 
  (e.g. CubeDM, Mango);
  however clients keep free to either process those host models as a
  whole or to chase individual components.
    \begin{itemize}
      \item Cross matching VOTables providing all the same e.g. the sky position serialization is easier.
            This also improves the reliability of the process since the engine does not need to infer information that is not in the FIELD meta-data.
      \item Building SEDs from datasets that have the same photometric calibration representation is straightforward.
   \end{itemize}          

  \item In more advanced cases, we need to be able to extract complete model instances from VOTables.
    \begin{itemize}
      \item Extracting  TimeSeries instances to feed a software built upon classes generated from that model.
      \item Building model instances that can be serialised in another format (e.g. json) in order to be shared in another context than 
            the VOTable processing (e.g. via SAMP).
   \end{itemize}         
    
   \item VOMAS can be used to annotate static science products such as e.g. time-series. It can also be used to annotate 
   on the fly TAP responses.
   It is to be noted and as shown by (ref- https://doi.org/10.48550/arXiv.2201.01732) that this process is not always 
   possible because the model compliance of the searched data strongly depends 
   on the query (missing searched columns, joins, data aggregation ...). 
   
    
\end{itemize} 

